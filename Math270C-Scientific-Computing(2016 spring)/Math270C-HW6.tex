\documentclass[12pt,a4paper]{article}
	%[fleqn] %%% --to make all equation left-algned--

\usepackage[top=1.2in, bottom=1.2in, left=0.7in, right=0.7in]{geometry}
%\usepackage{fullpage}

\usepackage{fancyhdr}\pagestyle{fancy}\rhead{Stephanie Wang}\lhead{Math270C - Homework 6}

\usepackage{pgfplots}\pgfplotsset{compat=1.12,colormap={mygreen}{rgb255(0cm)=(255,255,255);rgb255(1cm)=(255,255,255)}
    }
\usepackage{amsmath,amssymb,amsthm,amsfonts,microtype,stmaryrd}
%{mathtools,wasysym,yhmath}

\usepackage{xcolor}
\definecolor{ForestGreen}{rgb}{0.0,0.61,0.33}
\newcommand{\blue}[1]{\textcolor{blue}{#1}}\newcommand{\red}[1]{\textcolor{red}{#1}}\newcommand{\gray}[1]{\textcolor{gray}{#1}}\newcommand{\fgreen}[1]{\textcolor{ForestGreen}{#1}}

\usepackage{mdframed}
	%\newtheorem{mdexample}{Example}
	%\definecolor{warmgreen}{rgb}{0.8,0.9,0.85}
	% --Example:
	% \begin{center}
	% \begin{minipage}{0.8,0.9,0.85\textwidth}
	% \begin{mdframed}[backgroundcolor=warmgreen, 
	% skipabove=4pt,skipbelow=4pt,hidealllines=true, 
	% topline=false,leftline=false,middlelinewidth=10pt, 
	% roundcorner=10pt] 
	%%%% --CONTENTS-- %%%%
	% \end{mdframed}\end{minipage}\end{center}	

\usepackage{graphicx}
\graphicspath{ {/Users/KoraJr/Programming_Codes/C_family/Math270C/Assign3/} }
	% --Example:
	% \includegraphics[scale=0.5]{picture name}
%\usepackage{caption} %%% --some awful package to make caption...

%\usepackage{hyperref}\hypersetup{linktocpage,colorlinks}\hypersetup{citecolor=black,filecolor=black,linkcolor=black,urlcolor=black}

%%% --Text Fonts
%\usepackage{times} %%% --Times New Roman for LaTeX
%\usepackage{fontspec}\setmainfont{Times New Roman} %%% --Times New Roman; XeLaTeX only

%%% --Math Fonts
%\renewcommand{\mbf}[1]{\mathbf{#1}} %%% --vector
%\newcommand{\ca}[1]{\mathcal{#1}} %%% --"bigO"
%\newcommand{\bb}[1]{\mathbb{#1}} %%% --"Natural, Real numbers"
%\newcommand{\rom}[1]{\romannumeral{#1}} %%% --Roman numbers

%%% --Quick Arrows
\newcommand{\ra}[1]{\ifnum #1=1\rightarrow\fi\ifnum #1=2\Rightarrow\fi}

\newcommand{\la}[1]{\ifnum #1=1 \leftarrow\fi}

%%% --Special Editor Config
\renewcommand{\ni}{\noindent}
\newcommand{\onum}[1]{\raisebox{.5pt}{\textcircled{\raisebox{-1pt} {#1}}}}
\newcommand{\bbu}{\blacktriangleright}
\newcommand{\wbu}{\vartriangleright}

\newcommand{\claim}[1]{\underline{``{#1}":}}
\newcommand{\prob}[1]{\bf {#1}}

\newcommand{\bgfl}{\begin{flalign*}}
\newcommand{\bga}{\begin{align*}}
\def\beq{\begin{equation}} \def\eeq{\end{equation}}

\renewcommand{\l}{\left}\renewcommand{\r}{\right}

\newcommand{\casebrak}[2]{\left \{ \begin{array}{l} {#1}\\{#2} \end{array} \right.}
\newcommand{\ttm}[4]{\l[\begin{array}{cc}{#1}&{#2}\\{#3}&{#4}\end{array}\r]} %two-by-two-matrix
\newcommand{\tv}[2]{\l[\begin{array}{c}{#1}\\{#2}\end{array}\r]}

\newcommand{\dps}{\displaystyle}

\let\italiccorrection=\/
\def\/{\ifmmode\expandafter\frac\else\italiccorrection\fi}


%%% --General Math Symbols
\newcommand{\bc}{\because}
\newcommand{\tf}{\therefore}
\newcommand{\SUM}[2]{\sum\limits_{#1}^{#2}}
\newcommand{\PROD}[2]{\prod\limits_{#1}^{#2}}
\newcommand{\CUP}[2]{\bigcup\limits_{#1}^{#2}}
\newcommand{\CAP}[2]{\bigcap\limits_{#1}^{#2}}
\newcommand{\SUP}[1]{\sup\limits_{#1}}

\renewcommand{\o}{\circ}
\newcommand{\x}{\times}
\newcommand{\ox}{\otimes}

%%% --Special Math Characters
\newcommand{\R}{\mathbb R}%Real number
\newcommand{\N}{\mathbb N}%Nature number
\newcommand{\Z}{\mathbb Z}
\newcommand{\C}{\mathbb C}
\newcommand{\F}{\mathbb F}
\renewcommand{\O}{\mathcal{O}}
\newcommand{\A}{\mathcal{A}}%measurable sets
\renewcommand{\P}{\mathcal{P}}%power set

%%% --REAL ANALYSIS Symbols
\newcommand{\INT}[2]{\int_{#1}^{#2}}
\newcommand{\pdiff}[2]{\frac{\partial{#1}}{\partial{#2}}}
\newcommand{\UPINT}{\bar\int}
\newcommand{\UPINTRd}{\overline{\int_{\bb R ^d}}}
\newcommand{\supp}{\text{supp}}

\newcommand{\leb}{\lambda^\ast} %%% --Lebesgue
\renewcommand{\H}[1]{{\cal H}^{#1}} %%% --Hausdorff
\newcommand{\B}{\mathcal{B}} %%% --Borel set
\newcommand{\cL}{\mathcal{L}}
\newcommand{\I}{\mathcal{I}} %%% --index set
\newcommand{\Supp}[1]{\text{Supp}\left({#1}\right)}

\def\Xint#1{\mathchoice
{\XXint\displaystyle\textstyle{#1}}%
{\XXint\textstyle\scriptstyle{#1}}%
{\XXint\scriptstyle\scriptscriptstyle{#1}}%
{\XXint\scriptscriptstyle\scriptscriptstyle{#1}}%
\!\int}
\def\XXint#1#2#3{{\setbox0=\hbox{$#1{#2#3}{\int}$ }
\vcenter{\hbox{$#2#3$ }}\kern-.6\wd0}}
\def\ddashint{\Xint=}
\def\dashint{\Xint-}

\def\vx{\mathbf x}
\def\lip{\left\langle}
\def\rip{\right\rangle}
%%%%%%%%%%%%%%%%%%%%%%%%%%%%%%%%%%%%%%%%%%%%%%%%%%%%%%%%%%%%%%%%%%%%%%%%%%%%%%%%%%%%%%%%%%%%%%%%%%%%%%%%%%%%%%%%%%%%%%%%%%%%%%%%%%%%%%%%%%%%%%%%%%%%%%%%%%%%%%%%%%%%%%%%%%%%%%%%%%%%%%%%%%%%%%%%%%%%%%
\begin{document}
\subsection*{[T1]} 
\claim{$\lip r_{k-1}, p_k\rip = \lip b, p_k\rip$} \\
Since $\forall j < k, \lip p_k, p_j \rip_A = 0$, $p_k$ is $A$-orthogonal to the entire subspace $\mbox{span}\{p_0, p_1, \cdots, p_{k-1}\}$. Now from the fact that $r_{k-1} = b-Ax_{k-1}$ where $x_{k-1} \in \mbox{span}\{p_0, p_1, \cdots, p_{k-1}\}$, 
$$\lip r_{k-1}, p_k \rip= \lip b, p_k \rip - \lip x_k, p_k\rip_A = \lip b, p_k\rip$$
This shall prove the first equality about $\alpha_k$. The second follows from the fact $\lip r_k, p_k\rip = \|r\|^2$ which follows from the fact that $\lip r_k, p_k\rip = 0$.  \\
\\
\claim{$- {{< \vec r_{k-1}, \, A \vec p_{k-1}>} \over {< \vec p_{k-1}, \, A \vec p_{k -1} > }} \, = \, {{< \vec r_{k-1}, \, \vec r_{k-1}>} \over {< \vec r_{k-2}, \, \vec r_{k-2} > }}$}\\
Since $r_{k-1} = r_{k-2} - \alpha_{k-1}Ap_{k-1}$, the numerator 
$$\lip r_{k-1}, Ap_{k-1}\rip = \lip r_{k-1}, \/{r_{k-2}-r_{k-1}}{\alpha_{k-1}}\rip = \/1{\alpha_{k-1}} (\lip r_{k-1}, r_{k-2}\rip -\|r_{k-1}\|^2)$$
The last term equals to $\/{-1}{\alpha_{k-1}}\|r_{k-1}\|^2$ since $\{r_k\}$ had been proven to be an orthogonal set. Combined with the formula from part (a), $\alpha_{k-1} = \/{\|r_{k-2}\|^2}{\lip p_{k-1}, Ap_{k-1}\rip}$, the whole expression becomes
\bga
-\frac{\lip r_{k-1}, Ap_{k-1}\rip}{\lip p_{k-1}, Ap_{k-1}\rip} 
& = \frac{\|r_{k-1}\|^2}{\alpha_{k-1} \lip p_{k-1}, Ap_{k-1}\rip} \\
& = \frac{\|r_{k-1}\|^2}{\|r_{k-2}\|^2} 
\end{align*}


\subsection*{[C1/2]}
Taking (temporarily) the conclusion from Assignment 4, $\omega = 0.8$ and 2 relaxation operations were applied at each level, we produce the following table that shows the time (in second) required to preform a Pre-Conditioned Conjugate Gradient (PCCG) with maximal Multi-Grid level (also the index of the finest grid) ranging from 5 to 7 and corresponding minimal Multi-Grid level ranging from 0 to 4, 5 or 6, respectively. The time required by un-pre-conditioned Conjugate Gradient was also attached for comparison. Note that since the tolerance was set to be the same as \texttt{1e-06}, the precision of the following test results was the same. 
\begin{center}
\begin{tabular}{c|cccccccc}
 & CG & 0 & 1 & 2 & 3 & 4 & 5 & 6 \\
 \hline
5 & 0.0335198 & \fgreen{0.0174213} & 0.0188152 & 0.0222806 & 0.0323056 & 0.0495342 &  & \\
6 & 0.277031 & \fgreen{0.0719253} & 0.0808236 & 0.0972585 & 0.136014 & 0.227178 & 0.371307 & \\
7 & 2.26964 & \fgreen{0.301327} & 0.343037 & 0.409965 & 0.563305 & 0.92027 & 1.69528 & 2.93866\\
\end{tabular}
\end{center}
Observe that despite that we have to perform more computation within one V-cycle with minimal level 0 at each three maximal level, the time required to perform PCCG is still smaller than those with a larger minimal level. Also notice that all these numbers are significantly smaller than that for un-pre-conditioned CG. 
This has explained that the method gains an efficiency boost when the pre-conditioner is applied at the finest level. Here's a similarly labeled table with the content replaced by iteration count:
\begin{center}
\begin{tabular}{c|cccccccc}
 & CG & 0 & 1 & 2 & 3 & 4 & 5 & 6 \\
 \hline
5 & 102 & 5 & 6 & 8 & 13 & 24 &  & \\
6 & 202 & 5 & 6 & 8 & 12 & 23 & 46 & \\
7 & 396 & 5 & 6 & 8 & 12 & 22 & 44 & 90\\
\end{tabular}
\end{center}
However, it's worth noting that the computational resources acquired for one iteration at max-min multi-grid level (0, 5) is different from that of one iteration without pre-conditioner or other min multi-grid level. 

\subsection*{[C3/4]}
Here's some precious test results from running the test. (takes 8 minutes with $N=1$ and when $N=5$......) The table shows the optimal parameter sets and the time using them. 
\begin{center}
The following is test result with $N=1$, slight error could be expected. \\
\begin{tabular}{|c|c|c|c|c|}
\hline
max level & min level & $\omega$ & relaxation & time \\
\hline
5 & 0 & 0.7 & 2 & 0.01781 \\
6 & 0 & 1 & 2 & 0.075065 \\
7 & 0 & 1 & 2 & 0.315877 \\
\hline
\end{tabular}\\
The following is the test results (from running overnight) with $N=5$\\
\begin{tabular}{|c|c|c|c|c|}
\hline
max level & min level & $\omega$ & relaxation & time \\
\hline
5 & 1 & 1 & 2 & 0.0181482\\
6 & 0 & 0.8 & 2 & 0.0817066\\
7 & 0 & 0.9 & 2 & 0.336129\\
\hline
\end{tabular}
\end{center}
I also collected the data when running with $N=5$. It'll be attached as \texttt{MultiParameterTest\_data.csv} since the \LaTeX package \texttt{tabular} doesn't support cross pages table.
From the above result we can conservatively conclude that agreeing with the conclusion from Assignment 4, $\omega = 0.8$, 2 relaxation operations and min Multi-Grid level 0 will be the optimal parameter on average. 
\end{document}