\documentclass[12pt,a4paper]{article}
	%[fleqn] %%% --to make all equation left-algned--

\usepackage[top=1.2in, bottom=1.2in, left=0.7in, right=0.7in]{geometry}
%\usepackage{fullpage}
%\usepackage{filecontents}

\usepackage{fancyhdr}\pagestyle{fancy}\rhead{Stephanie Wang}\lhead{Math270C - Homework 4}

\usepackage{pgfplots}\pgfplotsset{compat=1.12,colormap={mygreen}{rgb255(0cm)=(255,255,255);rgb255(1cm)=(255,255,255)}
    }
\usepackage{amsmath,amssymb,amsthm,amsfonts,microtype,stmaryrd}
%{mathtools,wasysym,yhmath}

\usepackage{xcolor}
\definecolor{ForestGreen}{rgb}{0.0,0.61,0.33}
\newcommand{\blue}[1]{\textcolor{blue}{#1}}\newcommand{\red}[1]{\textcolor{red}{#1}}\newcommand{\gray}[1]{\textcolor{gray}{#1}}\newcommand{\fgreen}[1]{\textcolor{ForestGreen}{#1}}

\usepackage{mdframed}
	%\newtheorem{mdexample}{Example}
	%\definecolor{warmgreen}{rgb}{0.8,0.9,0.85}
	% --Example:
	% \begin{center}
	% \begin{minipage}{0.8,0.9,0.85\textwidth}
	% \begin{mdframed}[backgroundcolor=warmgreen, 
	% skipabove=4pt,skipbelow=4pt,hidealllines=true, 
	% topline=false,leftline=false,middlelinewidth=10pt, 
	% roundcorner=10pt] 
	%%%% --CONTENTS-- %%%%
	% \end{mdframed}\end{minipage}\end{center}	

\usepackage{graphicx}
\graphicspath{ {/Users/KoraJr/Programming_Codes/C_family/Math270C/Assign4/} }
	% --Example:
	% \includegraphics[scale=0.5]{picture name}
%\usepackage{caption} %%% --some awful package to make caption...

%\usepackage{hyperref}\hypersetup{linktocpage,colorlinks}\hypersetup{citecolor=black,filecolor=black,linkcolor=black,urlcolor=black}

%%% --Text Fonts
%\usepackage{times} %%% --Times New Roman for LaTeX
%\usepackage{fontspec}\setmainfont{Times New Roman} %%% --Times New Roman; XeLaTeX only

%%% --Math Fonts
%\renewcommand{\mbf}[1]{\mathbf{#1}} %%% --vector
%\newcommand{\ca}[1]{\mathcal{#1}} %%% --"bigO"
%\newcommand{\bb}[1]{\mathbb{#1}} %%% --"Natural, Real numbers"
%\newcommand{\rom}[1]{\romannumeral{#1}} %%% --Roman numbers

%%% --Quick Arrows
\newcommand{\ra}[1]{\ifnum #1=1\rightarrow\fi\ifnum #1=2\Rightarrow\fi}

\newcommand{\la}[1]{\ifnum #1=1 \leftarrow\fi}

%%% --Special Editor Config
\renewcommand{\ni}{\noindent}
\newcommand{\onum}[1]{\raisebox{.5pt}{\textcircled{\raisebox{-1pt} {#1}}}}
\newcommand{\bbu}{\blacktriangleright}
\newcommand{\wbu}{\vartriangleright}

\newcommand{\claim}[1]{\underline{``{#1}":}}
\newcommand{\prob}[1]{\bf {#1}}

\newcommand{\bgfl}{\begin{flalign*}}
\newcommand{\bga}{\begin{align*}}
\def\beq{\begin{equation}} \def\eeq{\end{equation}}

\renewcommand{\l}{\left}\renewcommand{\r}{\right}

\newcommand{\casebrak}[2]{\left \{ \begin{array}{l} {#1}\\{#2} \end{array} \right.}
\newcommand{\ttm}[4]{\l[\begin{array}{cc}{#1}&{#2}\\{#3}&{#4}\end{array}\r]} %two-by-two-matrix
\newcommand{\tv}[2]{\l[\begin{array}{c}{#1}\\{#2}\end{array}\r]}

\newcommand{\dps}{\displaystyle}

\let\italiccorrection=\/
\def\/{\ifmmode\expandafter\frac\else\italiccorrection\fi}


%%% --General Math Symbols
\newcommand{\bc}{\because}
\newcommand{\tf}{\therefore}
\newcommand{\SUM}[2]{\sum\limits_{#1}^{#2}}
\newcommand{\PROD}[2]{\prod\limits_{#1}^{#2}}
\newcommand{\CUP}[2]{\bigcup\limits_{#1}^{#2}}
\newcommand{\CAP}[2]{\bigcap\limits_{#1}^{#2}}
\newcommand{\SUP}[1]{\sup\limits_{#1}}

\renewcommand{\o}{\circ}
\newcommand{\x}{\times}
\newcommand{\ox}{\otimes}

%%% --Special Math Characters
\newcommand{\R}{\mathbb R}%Real number
\newcommand{\N}{\mathbb N}%Nature number
\newcommand{\Z}{\mathbb Z}
\newcommand{\C}{\mathbb C}
\newcommand{\F}{\mathbb F}
\renewcommand{\O}{\mathcal{O}}
\newcommand{\A}{\mathcal{A}}%measurable sets
\renewcommand{\P}{\mathcal{P}}%power set

%%% --REAL ANALYSIS Symbols
\newcommand{\INT}[2]{\int_{#1}^{#2}}
\newcommand{\pdiff}[2]{\frac{\partial{#1}}{\partial{#2}}}
\newcommand{\UPINT}{\bar\int}
\newcommand{\UPINTRd}{\overline{\int_{\bb R ^d}}}
\newcommand{\supp}{\text{supp}}

\newcommand{\leb}{\lambda^\ast} %%% --Lebesgue
\renewcommand{\H}[1]{{\cal H}^{#1}} %%% --Hausdorff
\newcommand{\B}{\mathcal{B}} %%% --Borel set
\newcommand{\cL}{\mathcal{L}}
\newcommand{\I}{\mathcal{I}} %%% --index set
\newcommand{\Supp}[1]{\text{Supp}\left({#1}\right)}

\def\Xint#1{\mathchoice
{\XXint\displaystyle\textstyle{#1}}%
{\XXint\textstyle\scriptstyle{#1}}%
{\XXint\scriptstyle\scriptscriptstyle{#1}}%
{\XXint\scriptscriptstyle\scriptscriptstyle{#1}}%
\!\int}
\def\XXint#1#2#3{{\setbox0=\hbox{$#1{#2#3}{\int}$ }
\vcenter{\hbox{$#2#3$ }}\kern-.6\wd0}}
\def\ddashint{\Xint=}
\def\dashint{\Xint-}

\def\vx{\mathbf x}

%%%%%%%%%%%%%%%%%%%%%%%%%%%%%%%%%%%%%%%%%%%%%%%%%%%%%%%%%%%%%%%%%%%%%%%%%%%%%%%%%%%%%%%%%%%%%%%%%%%%%%%%%%%%%%%%%%%%%%%%%%%%%%%%%%%%%%%%%%%%%%%%%%%%%%%%%%%%%%%%%%%%%%%%%%%%%%%%%%%%%%%%%%%%%%%%%%%%%%
\begin{document}
\subsection*{[T1]} 
(a) Inspect the recursive definition. For $r = 0$, 
$$\forall b\in \R^n, \tilde H(0)b = x^0 = 0$$
therefore $\tilde H(0) = O$ is the zero operator; for $r = 1,2, \cdots $
$$\forall b\in\R^n, \tilde H(r)b = P^{-1}b + P^{-1}Q\tilde H(r-1)b = (P^{-1} + P^{-1}Q\tilde H(r-1))b$$
$$\tilde H(r) = P^{-1} + P^{-1}Q \tilde H(r-1)$$
therefore, with the inductive hypothesis that $\tilde H(r-1)$ is a symmetric operator, so is $\tilde H(r)$.\\
\\
(b) For weighted Gauss-Jacobi, the iterative step is defined by
$$x^{k} = \omega D^{-1}(b - Rx^{k-1}) + (1-\omega)x^{k-1}$$
where $D$ consists of the diagonal entries of $A$ and $R$ the negative of all off diagonal entries and $\omega$ the relaxation parameter; therefore, the recursive definition for $\tilde H$ is $\tilde H(0) = O$, 
$$\tilde H(r) = \omega D^{-1} + ((1-\omega)I -  \omega D^{-1} R)\tilde H(r-1), r =  1,2, \cdots$$
Expand the recursive definition we get
\bga
\tilde H(r)
& = \SUM{k=0}{r-1} ((1-\omega)I -  \omega D^{-1} R)^k \omega D^{-1}  \\
& \ra1 \omega D^{-1} (I-((1-\omega)I -  \omega D^{-1} R))^{-1} \;\;\;\;\;\;(\text{as } r\ra1 \infty)\\
& = \omega D^{-1} (\omega I - \omega D^{-1}R)^{-1} \\
& = (D-R)^{-1} = A^{-1}
\end{align*}
Here we relied on one assumption that $\rho((1-\omega)I -  \omega D^{-1} R) < 1$ for the convergence of the power series. This conclusion confirms that weighted Gauss-Jacobi solves the linear problem $Ax = b$. 
\subsection*{[T2]}
(a) In this case, we will need to store interior grids of size $(2^1-1)\x (2^1-1), (2^2-1)\x(2^2-1), \cdots, (2^{K+1}-1)\x(2^{K+1}-1)$; summing these numbers up we get 
\bga
\SUM{k=1}{K+1} (2^{k}-1)^2 
& = \SUM{k=1}{K+1} 2^{2k} - 2\cdot 2^k + 1\\
& =\/{4(4^{K+1}-1)}{4-1} - 2 \/{2(2^{K+1}-1)}{2-1} + K+1 \\
& = \/43(4^{K+1}-1) - 4(2^{K+1}-1) + K+1 \\
& = \l(\/13 \cdot 2^{K+1}- 1\r)2^{K+3} + K+ \/{11}3
\end{align*}
(some number that doesn't look so good...) \\
\\
(b) Assume 1 relaxation is applied at each level; (simply multiply the final result with the number of relaxation will give the desired answer in other cases) since in one single V cycle we go from level $K$ down to level $0$, (note no relaxation operation is applied at level $0$) and back to level $K$, in total we will need to apply $2K$ relaxation operations. (note: the size of the relaxation operations are different among different level!) \\
\\
(c) Let $\gamma$ be the number of relaxation operation applied at each level. Since the relaxation operation is a pentadiagonal operator, for a fixed size grid, the arithmetic operation per each grid node is $\O(1)$. \fgreen{(ignore the edge cases for all good! )}; therefore, in order to compare the numerical resources used, we need only compare the number of total grid nodes involved. The number of grid nodes involved for a V cycle is (number taken from part (a)), assuming $\gamma$ relaxation operations are applied at each level:
$$\gamma \SUM{k=2}{K+1} (2^k -1)^2 = \gamma\l(\/13 \cdot 2^{K+1} - 1\r)2^{K+3} + K\gamma + \/83\gamma $$
(Note that no operation is done at the lowest level, so the summation starts from $k=2$) On the other hand, the number of grid nodes involved for one relaxation operation applied on the finest grid is $(2^{K+1}-1)^2 = 4^{K+1} - 2^{K+2} + 1$, and the ratio between these two is 
\bga
\frac{\/13\gamma\l(2^{K+1} - 3\r)2^{K+3} + K\gamma + \/83\gamma}{(2^{K+1}-1)^2} 
\end{align*}
Something you wouldn't really want to evaluate... but APPROXIMATELY, assuming $K$ is large, is $\/43\gamma$. 


\subsection*{[C2]}
The following is the results of running my program. 
\begin{verbatim}
XXXX MultiGrid2d XXXX

Vcycle : 1Residual Norm : 0.78113
Vcycle : 2Residual Norm : 0.154505
Vcycle : 3Residual Norm : 0.0305834
Vcycle : 4Residual Norm : 0.0060601
Vcycle : 5Residual Norm : 0.0012019
Vcycle : 6Residual Norm : 0.000238576

XXXX MultiGrid2d Test Output XXXX
X-Panel Count    : 64
Y-Panel Count 	  : 64
X-Wavenumber     : 1
Y-Wavenumber     : 1
Omega            : 0.75
RelaxationCount  : 2
MultiGrid V-cycles       = 6
Residual norm (L2)       = 0.000238576
Residual norm (Inf)      = 6.90996e-05
\end{verbatim}
\subsection*{[C3]}
Here's the table of the V-cycle count times relaxation count (this number multiplied by $\/43$ will be equivalent to the number of relaxation applied at finest level, see [T2] part (c)) with $\omega = 0.1, 0.2, \cdots, 0.9$ (shown along the vertical axis) and relaxation count varying from $2$ to $6$ (shown along the horizontal axis). See the minimum appears at $\omega = 0.7, 0.8$ with relaxation count equals $2$.
\begin{center}
\begin{tabular}{c|ccccc}
&2&3&4&5&6\\
\hline
0.1&146&147&148&150&150\\
0.2&72&72&76&75&78\\
0.3&48&48&52&55&60\\
0.4&36&36&40&45&48\\
0.5&28&30&36&40&42\\
0.6&24&27&32&35&42\\
0.7&20&27&32&35&42\\
0.8&20&24&28&35&36\\
0.9&22&24&28&30&36
\end{tabular}
\end{center}
Here's one more set of data generated with max multi-grid level 7; see the minimum also appears around $\omega = 0.7, 0.8$ with 2 relaxations.
\begin{center}
\begin{tabular}{c|ccccc}
&2&3&4&5&6\\
\hline
0.1&160&159&160&160&162\\
0.2&78&78&80&80&84\\
0.3&52&54&56&55&60\\
0.4&38&39&44&45&54\\
0.5&30&33&36&40&48\\
0.6&26&30&32&40&42\\
0.7&22&27&32&35&42\\
0.8&20&24&32&35&42\\
0.9&20&24&28&35&36
\end{tabular}
\end{center}



\end{document}

\begin{tikzpicture}
\begin{axis}[enlargelimits=false,colorbar]
\addplot[matrix plot,
nodes near coords=\coordindex,mark=*,
point meta=explicit]
coordinates {
(0,0) [0] (1,0) [1] (2,0) [2]
(0,1) [3] (1,1) [4] (2,1) [5]
(0,2) [6] (1,2) [7] (2,2) [8]
};
\end{axis}
\end{tikzpicture}


\begin{tikzpicture}
\begin{axis}
\addplot3 [surf, xlabel=$\omega$, ylabel=relaxation count, title=total relaxation taken]
  table[x=Omega, y=RelaxationSetting, z=RelaxationCount, col sep=comma] {/Users/KoraJr/Programming_Codes/C_family/Math270C/Assign4/MultiGrid2d_Results.csv};
\end{axis}
\end{tikzpicture}
