\documentclass[12pt,a4paper]{article}
	%[fleqn]		%% --to make all equation left-algned
\usepackage[top=2in, bottom=1in, left=1in, right=1in]{geometry}
%\usepackage{fullpage}

\usepackage{fancyhdr}\pagestyle{fancy}\rhead{Stephanie Wang}\lhead{Math270B Homework 1}

\usepackage{amsmath,amssymb,amsthm,amsfonts,microtype,stmaryrd}
	%{mathtools,wasysym,yhmath}

\usepackage[usenames,dvipsnames]{xcolor}\newcommand{\blue}[1]{\textcolor{blue}{#1}}\newcommand{\red}[1]{\textcolor{red}{#1}}\newcommand{\gray}[1]{\textcolor{gray}{#1}}
\newcommand{\fgreen}[1]{\textcolor{ForestGreen}{#1}}

%\usepackage{mdframed}
	%\newtheorem{mdexample}{Example}
	%\definecolor{warmgreen}{rgb}{0.8,0.9,0.85}
	% --Example:
	% \begin{center}
	% \begin{minipage}{0.7\textwidth}
	% \begin{mdframed}[backgroundcolor=warmgreen, 
	% skipabove=4pt,skipbelow=4pt,hidealllines=true, 
	% topline=false,leftline=false,middlelinewidth=10pt, 
	% roundcorner=10pt] 
	%%%% --CONTENTS-- %%%%
	% \end{mdframed}\end{minipage}\end{center}	

%\usepackage{graphicx} \graphicspath{{}}
	% --Example:
	% \includegraphics[scale=0.5]{picture name}

%\usepackage{hyperref}\hypersetup{linktocpage,colorlinks}\hypersetup{citecolor=black,filecolor=black,linkcolor=black,urlcolor=black,breaklinks=true}

%%% --Text Fonts
%\usepackage{times} %%% --Times New Roman for LaTeX
%\usepackage{fontspec}\setmainfont{Times New Roman} %%% --Times New Roman; XeLaTeX only

%%% --Quick Arrows
\newcommand{\ra}[1]{\ifnum #1=1\rightarrow\fi\ifnum #1=2\Rightarrow\fi\ifnum #1=3\Rrightarrow\fi\ifnum #1=4\rightrightarrows\fi\ifnum #1=5\rightleftarrows\fi\ifnum #1=6\mapsto\fi\ifnum #1=7\iffalse\fi\fi\ifnum #1=8\twoheadrightarrow\fi\ifnum #1=9\rightharpoonup\fi\ifnum #1=0\rightharpoondown\fi}
%\newcommand{\la}[1]{\ifnum #1=1\leftarrow\fi\ifnum #1=2\Leftarrow\fi\ifnum #1=3\Lleftarrow\fi\ifnum #1=4\leftleftarrows\fi\ifnum #1=5\rightleftarrows\fi\ifnum #1=6\mapsfrom\ifnum #1=7\iffalse\fi\fi\ifnum #1=8\twoheadleftarrow\fi\ifnum #1=9\leftharpoonup\fi\ifnum #1=0\leftharpoondown\fi}
%\newcommand{\ua}[1]{\ifnum #1=1\uparrow\fi\ifnum #1=2\Uparrow\fi}
%\newcommand{\da}[1]{\ifnum #1=1\downarrow\fi\ifnum #1=2\Downarrow\fi}

%%% --Quick Editor Config
\newcommand{\nid}{\noindent}
\newcommand{\dps}{\displaystyle}

\newcommand{\onum}[1]{\raisebox{.5pt}{\textcircled{\raisebox{-1pt} {#1}}}}
\newcommand{\claim}[1]{\underline{``{#1}":}}
\newcommand{\prob}[1]{{\bf {#1}}}

%%% --Quick Math Mode Config
\renewcommand{\l}{\left}\renewcommand{\r}{\right}
\newcommand{\casebrak}[2]{\left \{ \begin{array}{l} {#1}\\{#2} \end{array} \right.}
\newcommand{\casedef}[4]{\left \{ \begin{array}{ll} {#1} & {#2}\\{#3} & {4} \end{array} \right.}
%\newcommand{\ttm}[4]{\l[\begin{array}{cc}{#1}&{#2}\\{#3}&{#4}\end{array}\r]} %two-by-two-matrix
%\newcommand{\tv}[2]{\l[\begin{array}{c}{#1}\\{#2}\end{array}\r]}
\let\italiccorrection=\/
\def\/{\ifmmode\expandafter\frac\else\italiccorrection\fi}

%%% --Special Math Characters
\def\R{\ifmmode\mathbb R\fi}
\def\N{\ifmmode\mathbb N\fi}
\let\slashedO=\O	% put the value of \O at this line to \slashedO
\def\O{\ifmmode\mathcal O\else\slashedO\fi}	% whenever \O is called it will evaluate the following expression
% although \newcommand does the same thing as \def, it throws error when overwriting existing command; 

%%% --General Math Symbols/Operators
\newcommand{\bc}{\because}
\newcommand{\tf}{\therefore}

\newcommand{\INT}[2]{\int_{#1}^{#2}}
\newcommand{\SUM}[2]{\sum\limits_{#1}^{#2}}
\newcommand{\PROD}[2]{\prod\limits_{#1}^{#2}}
\newcommand{\CUP}[2]{\bigcup\limits_{#1}^{#2}}
\newcommand{\CAP}[2]{\bigcap\limits_{#1}^{#2}}
\newcommand{\SUP}[1]{\sup\limits_{#1}}

%\renewcommand{\o}{\circ}
%\newcommand{\x}{\times}
%\newcommand{\ox}{\otimes}

%%% --Analysis Symbols
\newcommand{\pd}[2]{\frac{\partial{#1}}{\partial{#2}}}
% \newcommand{\UPINT}{\bar\int}
% \newcommand{\UPINTRd}{\overline{\int_{\bb R ^d}}}
% \newcommand{\SUP}[1]{\sup\limits_{#1}}
% \newcommand{\INF}[1]{\inf\limits_{#1}}

%%% --Numericals Symbols
\DeclareMathOperator*{\argmin}{arg\,min}
\DeclareMathOperator*{\argmax}{arg\,max}
\newcommand{\dt}{\Delta t}
%\newcommand{\nxt}{^{n+1}}
%\newcommand{\pvs}{^{n-1}}
%\newcommand{\hfnxt}{^{n+\frac12}}

%%% --Matrix Analysis Symbols
\def\tr{\text{tr}}
\def\vA{\mathbf{A}}
\def\vB{\mathbf{B}}\def\cB{\mathcal{B}}
\def\vC{\mathbf{C}}
\def\vD{\mathbf{D}}
\def\vE{\mathbf{E}}
\def\vF{\mathbf{F}}\def\tvF{\tilde{\mathbf{F}}}
\def\vG{\mathbf{G}}
\def\vH{\mathbf{H}}
\def\vI{\mathbf{I}}\def\cI{\mathcal{I}}
\def\vJ{\mathbf{J}}
\def\vK{\mathbf{K}}
\def\vL{\mathbf{L}}\def\cL{\mathcal{L}}
\def\vM{\mathbf{M}}
\def\vN{\mathbf{N}}\def\cN{\mathcal{N}}
\def\vO{\mathbf{O}}
\def\vP{\mathbf{P}}
\def\vQ{\mathbf{Q}}
\def\vR{\mathbf{R}}
\def\vS{\mathbf{S}}
\def\vT{\mathbf{T}}
\def\vU{\mathbf{U}}
\def\vV{\mathbf{V}}
\def\vW{\mathbf{W}}
\def\vX{\mathbf{X}}
\def\vY{\mathbf{Y}}
\def\vZ{\mathbf{Z}}

\def\va{\mathbf{a}}
\def\vb{\mathbf{b}}
\def\vc{\mathbf{c}}
\def\vd{\mathbf{d}}
\def\ve{\mathbf{e}}
\def\vf{\mathbf{f}}
\def\vg{\mathbf{g}}
\def\vh{\mathbf{h}}
\def\vi{\mathbf{i}}
\def\vj{\mathbf{j}}
\def\vk{\mathbf{k}}
\def\vl{\mathbf{l}}
\def\vm{\mathbf{m}}
\def\vn{\mathbf{n}}
\def\vo{\mathbf{o}}
\def\vp{\mathbf{p}}
\def\vq{\mathbf{q}}
\def\vr{\mathbf{r}}
\def\vs{\mathbf{s}}
\def\vt{\mathbf{t}}
\def\vu{\mathbf{u}}
\def\vv{\mathbf{v}}\def\tvv{\tilde{\mathbf{v}}}
\def\vw{\mathbf{w}}
\def\vx{\mathbf{x}}\def\tvx{\tilde{\mathbf{x}}}
\def\vy{\mathbf{y}}
\def\vz{\mathbf{z}}

\def\odiv{\fgreen{\div}}
\newcommand{\fl}{\mbox{fl}}
%%%%%%%%%%%%%%%%%%%%%%%%%%%%%%%%%%%%%%%%%%%%%%%%%%%%%%%%%%%%%%%%%%%%%%%%%%%%%%%%%%%%%%%%%%%%%%%%%%%%%%%%%%%%%%%%%%%%%%%%%%%%%%%%%%%%%%%%%%%%%%%%%%%%%%%%%%%%%%%%%%%%%%%%%%%%%%%%%%%%%%%%%%%%%%%%%%%%%%
\begin{document}
\subsubsection*{Problem 1}
Fix $x, y\in\R$. Consider $\tilde y = \fl(y) \in \R$, we proceed to find $\tilde x \in \R$ for backward stability, {\it i.e.},
$$\fl(\fl(x)\div \fl(y)) = \tilde x \div \tilde y = \tilde x \div \fl(y)$$
Let $\tilde x = \fl(\fl(x)\div\fl(y))\fl(y)$; it remains to show that $\/{|x-\tilde x|}{|x|} \leq \epsilon$ for some $\epsilon > 0$. Assuming $\fl(y) \neq 0$, 
\begin{align*}
\epsilon_{machine} &\geq 
\frac{|\fl(\fl(x)\div \fl(y)) - \fl(x)\div\fl(y)|}{|\fl(x)\div\fl(y)|} \\
&= \frac{|\fl(\fl(x)\div \fl(y))\fl(y) - \fl(x)|}{|\fl(x)|} \\
&= \frac{|\tilde x - \fl(x)|}{|\fl(x)|}
\end{align*}
Now since $\displaystyle\/{|x-\fl(x)|}{|x|}\leq \epsilon_{machine}$,
\begin{align*}
\frac{|x-\tilde x|}{|x|} &\leq \/{|x-\fl(x)|}{|x|} + \/{|\fl(x) - \tilde x|}{|x|} \\
&\leq \epsilon_{machine} + \epsilon_{machine}\/{|\fl(x)|}{|x|} \\
&\leq \epsilon_{machine} + \epsilon_{machine}\/{|x| + \epsilon_{machine}|x|}{|x|} \\
&\leq \epsilon_{machine}\l(2 + \epsilon_{machine}\r)
\end{align*} \qed

\newpage\subsubsection*{Problem 2}
All single-precision floating point numbers has the form
$$x = (-1)^s 2^E(1+f)$$
where $s = 0, 1$ determines the sign of $x$, $E$ ranges from $-127$ to $128$, and $f = \SUM{j=1}{23} f_j 2^{-j}$, or $f = 0.f_1f_2f_3\cdots f_{23}$ in binary representation. ($f_j$'s are $0$ or $1$.) Surely we see if $s=0$, $E=128$, and all $f_j$'s are all $1$, then the largest possible single-precision floating point number
\begin{align*}
x_{max} &= (-1)^0 2^{128} \l(1 + \SUM{j=1}{23} 2^{-j}\r)\\
&= \SUM{j=0}{23} 2^{128-j}  = \SUM{k=105}{128} 2^k
\end{align*}
is a sum of positive integers therefore itself is an integer; however, this does not capture the ``holes" of the floating point number system. Observe for $E \geq 23$, 
\begin{align*}
x &= (-1)^0 2^{E} \l(1 + \SUM{j=1}{23} f_j 2^{-j}\r)\\
&= 2^E + \SUM{j=1}{23} f_j2^{E-j}
\end{align*}
is not much but sum of integers. Consider 
\begin{align*}
z &= (-1)^0 2^{24}\l(1 + \SUM{j=1}{24} 2^{-j}\r) \\
&= \SUM{k=0}{24}2^k \\
&= \underbrace{111 \; \cdots \; 1} \\
& \;\;\;\;\mbox{(25-digits)} \\
\end{align*}
This number can't be represented in single-precision floating point number system since it has more nontrivial digits than 23.  \qed 



\newpage\subsubsection*{Problem 3, 4}
Solutions to both problem 3 and 4 are attached in the codes below. 
\begin{verbatim}
function hw1
%% build sample matrix A
n = 10;
A = zeros(n);
A(1,1) = 2;
A(1,2) = -1;
for i=2:n-1
    A(i,i-1) = -1;
    A(i,i) = 2;
    A(i,i+1) = -1;
end
A(n,n-1) = -1;
A(n,n) = 2;

%% QR
[Q, R] = myhouseholderQR(A);
resQR = A - Q*R;
format shortEng;
fprintf('Residual of the Householder QR = \n');
disp(norm(resQR(:)));

%% LU
[L,U] = myLU(A);
resLU = A - L*U;
format shortEng;
fprintf('Residual of LU = \n');
disp(norm(resLU(:)));
end

function [Q, R] = myhouseholderQR(A)
% [Q, R] = householderQR(A)
% Computes the QR-decomposition of input matrix A
[m,n] = size(A);
Q = eye(m);
for j=1:n-1
    x = A(j:m, j);
    v = [zeros(j-1, 1); x + norm(x) * eye(m-j+1,1)];
    Pv = eye(m) - (2*v)*v'/(v'*v);
    A = Pv * A;
    Q = Pv * Q;
end
Q = Q';
R = A;
end

function [L, U] = myLU(A)
[m,n] = size(A);
assert(m==n, 'Input is not a square matrix.\n');
clear m;

L = eye(n);
U = A;
for j=1:n-1
    for i=j+1:n
        factor = U(i,j) / U(j,j);
        U(i,:) = U(i,:) - factor * U(j,:);
        L(i,:) = L(i,:) - factor * L(j,:);
    end
end
L = inv(L);
end
\end{verbatim}



\end{document}











