\documentclass[12pt,a4paper]{article}
	%[fleqn] %%% --to make all equation left-algned--

\usepackage[top=1.2in, bottom=1.2in, left=0.7in, right=0.7in]{geometry}
%\usepackage{fullpage}

\usepackage{fancyhdr}\pagestyle{fancy}\rhead{Stephanie Wang}\lhead{Math270C - Homework 1}

\usepackage{amsmath,amssymb,amsthm,amsfonts,microtype,stmaryrd}
%{mathtools,wasysym,yhmath}

\usepackage[usenames,dvipsnames]{xcolor}\newcommand{\blue}[1]{\textcolor{blue}{#1}}\newcommand{\red}[1]{\textcolor{red}{#1}}\newcommand{\gray}[1]{\textcolor{gray}{#1}}
\newcommand{\fgreen}[1]{\textcolor{ForestGreen}{#1}}

\usepackage{mdframed}
	%\newtheorem{mdexample}{Example}
	%\definecolor{warmgreen}{rgb}{0.8,0.9,0.85}
	% --Example:
	% \begin{center}
	% \begin{minipage}{0.8,0.9,0.85\textwidth}
	% \begin{mdframed}[backgroundcolor=warmgreen, 
	% skipabove=4pt,skipbelow=4pt,hidealllines=true, 
	% topline=false,leftline=false,middlelinewidth=10pt, 
	% roundcorner=10pt] 
	%%%% --CONTENTS-- %%%%
	% \end{mdframed}\end{minipage}\end{center}	

\usepackage{graphicx}
\graphicspath{ {/Users/KoraJr/Documents/MATLAB} }
	% --Example:
	% \includegraphics[scale=0.5]{picture name}
%\usepackage{caption} %%% --some awful package to make caption...

%\usepackage{hyperref}\hypersetup{linktocpage,colorlinks}\hypersetup{citecolor=black,filecolor=black,linkcolor=black,urlcolor=black}

%%% --Text Fonts
%\usepackage{times} %%% --Times New Roman for LaTeX
%\usepackage{fontspec}\setmainfont{Times New Roman} %%% --Times New Roman; XeLaTeX only

%%% --Math Fonts
%\renewcommand{\mbf}[1]{\mathbf{#1}} %%% --vector
%\newcommand{\ca}[1]{\mathcal{#1}} %%% --"bigO"
%\newcommand{\bb}[1]{\mathbb{#1}} %%% --"Natural, Real numbers"
%\newcommand{\rom}[1]{\romannumeral{#1}} %%% --Roman numbers

%%% --Quick Arrows
\newcommand{\ra}[1]{\ifnum #1=1\rightarrow\fi\ifnum #1=2\Rightarrow\fi}

\newcommand{\la}[1]{\ifnum #1=1 \leftarrow\fi}

%%% --Special Editor Config
\renewcommand{\ni}{\noindent}
\newcommand{\onum}[1]{\raisebox{.5pt}{\textcircled{\raisebox{-1pt} {#1}}}}
\newcommand{\bbu}{\blacktriangleright}
\newcommand{\wbu}{\vartriangleright}

\newcommand{\claim}[1]{\underline{``{#1}":}}
\newcommand{\prob}[1]{\bf {#1}}

\newcommand{\bgfl}{\begin{flalign*}}
\newcommand{\bga}{\begin{align*}}
\def\beq{\begin{equation}} \def\eeq{\end{equation}}

\renewcommand{\l}{\left}\renewcommand{\r}{\right}

\newcommand{\casebrak}[2]{\left \{ \begin{array}{l} {#1}\\{#2} \end{array} \right.}
%\newcommand{\ttm}[4]{\l[\begin{array}{cc}{#1}&{#2}\\{#3}&{#4}\end{array}\r]} %two-by-two-matrix
%\newcommand{\tv}[2]{\l[\begin{array}{c}{#1}\\{#2}\end{array}\r]}

\newcommand{\dps}{\displaystyle}

\let\italiccorrection=\/
\def\/{\ifmmode\expandafter\frac\else\italiccorrection\fi}


%%% --General Math Symbols
\newcommand{\bc}{\because}
\newcommand{\tf}{\therefore}
\newcommand{\SUM}[2]{\sum\limits_{#1}^{#2}}
\newcommand{\PROD}[2]{\prod\limits_{#1}^{#2}}
\newcommand{\CUP}[2]{\bigcup\limits_{#1}^{#2}}
\newcommand{\CAP}[2]{\bigcap\limits_{#1}^{#2}}
\newcommand{\SUP}[1]{\sup\limits_{#1}}

\renewcommand{\o}{\circ}
\newcommand{\x}{\times}
\newcommand{\ox}{\otimes}

%%% --Special Math Characters
\newcommand{\R}{\mathbb R}%Real number
\newcommand{\N}{\mathbb N}%Nature number
\newcommand{\Z}{\mathbb Z}
\newcommand{\C}{\mathbb C}
\newcommand{\F}{\mathbb F}
\renewcommand{\O}{\mathcal{O}}
\newcommand{\A}{\mathcal{A}}%measurable sets
\renewcommand{\P}{\mathcal{P}}%power set

%%% --REAL ANALYSIS Symbols
\newcommand{\INT}[2]{\int_{#1}^{#2}}
\newcommand{\pdiff}[2]{\frac{\partial{#1}}{\partial{#2}}}
\newcommand{\UPINT}{\bar\int}
\newcommand{\UPINTRd}{\overline{\int_{\bb R ^d}}}
\newcommand{\supp}{\text{supp}}

\newcommand{\leb}{\lambda^\ast} %%% --Lebesgue
\renewcommand{\H}[1]{{\cal H}^{#1}} %%% --Hausdorff
\newcommand{\B}{\mathcal{B}} %%% --Borel set
\newcommand{\cL}{\mathcal{L}}
\newcommand{\I}{\mathcal{I}} %%% --index set
\newcommand{\Supp}[1]{\text{Supp}\left({#1}\right)}

\def\Xint#1{\mathchoice
{\XXint\displaystyle\textstyle{#1}}%
{\XXint\textstyle\scriptstyle{#1}}%
{\XXint\scriptstyle\scriptscriptstyle{#1}}%
{\XXint\scriptscriptstyle\scriptscriptstyle{#1}}%
\!\int}
\def\XXint#1#2#3{{\setbox0=\hbox{$#1{#2#3}{\int}$ }
\vcenter{\hbox{$#2#3$ }}\kern-.6\wd0}}
\def\ddashint{\Xint=}
\def\dashint{\Xint-}

\def\vx{\mathbf x}

%%%%%%%%%%%%%%%%%%%%%%%%%%%%%%%%%%%%%%%%%%%%%%%%%%%%%%%%%%%%%%%%%%%%%%%%%%%%%%%%%%%%%%%%%%%%%%%%%%%%%%%%%%%%%%%%%%%%%%%%%%%%%%%%%%%%%%%%%%%%%%%%%%%%%%%%%%%%%%%%%%%%%%%%%%%%%%%%%%%%%%%%%%%%%%%%%%%%%%
\begin{document}
\subsection*{[T1]}
(a/b) Since discretization fails for the end points, the matrix will be of size $nx\x nx$. Then this matrix $A = [a_{ij}]$ is given by 
$$a_{1j} = \l\{\begin{array}{ll}
-2/hx^2, &j=i\\
1/hx^2, &j=i+1\\
0, &\text{else}
\end{array}\r.$$
$$2 \leq i \leq nx-1 \ra2 a_{ij} = \l\{\begin{array}{ll}
1/hx^2, &j=i-1\\
-2/hx^2, &j=i\\
1/hx^2, &j=i+1\\
0, &\text{else}
\end{array}\r.$$
$$a_{nx,j} = \l\{\begin{array}{ll}
1/hx^2, &j=i-1\\
-2/hx^2, &j=i\\
0, &\text{else}
\end{array}\r.$$
(the indices used here are 1-based, despite ridicule, or inconsistency with usage in C++) and the bandwidth is \underline{three}; matrix is tridiagonal. \\
\\
(c) Let $\vx = [\sin(k\pi i hx)] \in \F^{nx}$, then
\bga
(A\vx)_i 
= & \frac1{hx^2}\l(\sin(k\pi (i+1)hx) - 2\sin(k\pi ihx) + \sin(k\pi(i-1)hx)\r)\\
= & \frac1{hx^2 }\l(  2\cos(k\pi (i+1/2) hx)\sin(k\pi (1/2) hx) - 2\cos(k\pi (i-1/2)hx)\sin(k\pi (1/2)hx)  \r)\\
= & \frac2{hx^2}\sin(k\pi(1/2)hx) \l(\cos(k\pi(i+1/2)hx) - \cos(k\pi(i-1/2)hx) \r) \\
= & \frac2{hx^2}\sin(k\pi(1/2)hx)(-2\sin(k\pi i hx)\sin(k\pi (1/2)hx)) \\
= & \frac{-4}{hx^2} \sin^2(k\pi (1/2)hx) \sin(k\pi i hx) \\
= & \l(\frac{-4\sin^2(k\pi (1/2)hx) }{hx^2}\r) (\vx)_i
\end{align*}
This holds for all $i$, so we see the eigenvalue is 
$$\lambda_k = \frac{-4\sin^2(k\pi (1/2)hx) }{hx^2}$$

\subsection*{[T2]}
(a) The matrix has three diagonals and one sup- and one sub-diagonal that are $nx$ entries away from the diagonal. The matrix $A = [a_{ij}]$ is given by 
$$a_{ij} = \l\{\begin{array}{ll}
1/hy^2, &j = i-nx\\
1/hx^2, &j = i-1\\
-2/hx^2-2/hx^2, &j=i\\
1/hx^2, &j=i+1\\
1/hy^2, & j = i+nx\\
0, &\text{else}
\end{array}\r.$$
Note that since boundary values were set to zero, in this mathematical definition when the index is invalid for grasping we just ignore the lines. \\
\\
(b) The dimension of the matrix is $(nx \x ny)\x (nx \x ny)$.  \\
\\
(c) For two wave number, since the operation is invariant with multiplying by a constant, we can in fact separate the operation on $x$-axis and $y$-axis; therefore,the eigenvalue of the 5-stencil discretization is the sum $\lambda_{k_1} + \lambda_{k_2}$. 


\subsection*{[C1]}
The results from running my program (also attached) are 
\begin{verbatim}
XXXX  Laplacian 2D Operator Test Output XXXX 
X-Panel Count : 100
Y-Panel Count : 100
X-Wavenumber  : 3
Y-Wavenumber  : 4
L_2    Error in operator = 6.79462e-11
L_Inf  Error in operator = 2.63753e-11

XXXX Laplacian 2D Matrix Test Output XXXX 
X-Panel Count : 100
Y-Panel Count : 100
X-Wavenumber  : 3
Y-Wavenumber  : 4
L_2    Error in operator = 6.87485e-11
L_Inf  Error in operator = 2.70006e-11

XXXX  Laplacian 2D Operator Test Output XXXX 
X-Panel Count : 50
Y-Panel Count : 100
X-Wavenumber  : 3
Y-Wavenumber  : 4
L_2    Error in operator = 5.39747e-11
L_Inf  Error in operator = 1.92415e-11

XXXX Laplacian 2D Matrix Test Output XXXX 
X-Panel Count : 50
Y-Panel Count : 100
X-Wavenumber  : 3
Y-Wavenumber  : 4
L_2    Error in operator = 5.45504e-11
L_Inf  Error in operator = 2.02363e-11
\end{verbatim}

\end{document}