\documentclass[12pt,a4paper]{article}
	%[fleqn] %%% --to make all equation left-algned--

\usepackage[top=1.2in, bottom=1.2in, left=0.7in, right=0.7in]{geometry}
%\usepackage{fullpage}

\usepackage{fancyhdr}\pagestyle{fancy}\rhead{Stephanie Wang}\lhead{Math270C - Homework 8}

\usepackage{pgfplots}\pgfplotsset{compat=1.12,colormap={mygreen}{rgb255(0cm)=(255,255,255);rgb255(1cm)=(255,255,255)}
    }
\usepackage{amsmath,amssymb,amsthm,amsfonts,microtype,stmaryrd}
%{mathtools,wasysym,yhmath}

\usepackage{xcolor}
\definecolor{ForestGreen}{rgb}{0.0,0.61,0.33}
\newcommand{\blue}[1]{\textcolor{blue}{#1}}\newcommand{\red}[1]{\textcolor{red}{#1}}\newcommand{\gray}[1]{\textcolor{gray}{#1}}\newcommand{\fgreen}[1]{\textcolor{ForestGreen}{#1}}

\usepackage{mdframed}
	%\newtheorem{mdexample}{Example}
	%\definecolor{warmgreen}{rgb}{0.8,0.9,0.85}
	% --Example:
	% \begin{center}
	% \begin{minipage}{0.8,0.9,0.85\textwidth}
	% \begin{mdframed}[backgroundcolor=warmgreen, 
	% skipabove=4pt,skipbelow=4pt,hidealllines=true, 
	% topline=false,leftline=false,middlelinewidth=10pt, 
	% roundcorner=10pt] 
	%%%% --CONTENTS-- %%%%
	% \end{mdframed}\end{minipage}\end{center}	

\usepackage{graphicx}
\graphicspath{ {/Users/KoraJr/Programming_Codes/C_family/Math270C/Assign3/} }
	% --Example:
	% \includegraphics[scale=0.5]{picture name}
%\usepackage{caption} %%% --some awful package to make caption...

%\usepackage{hyperref}\hypersetup{linktocpage,colorlinks}\hypersetup{citecolor=black,filecolor=black,linkcolor=black,urlcolor=black}

%%% --Text Fonts
%\usepackage{times} %%% --Times New Roman for LaTeX
%\usepackage{fontspec}\setmainfont{Times New Roman} %%% --Times New Roman; XeLaTeX only

%%% --Math Fonts
%\renewcommand{\mbf}[1]{\mathbf{#1}} %%% --vector
%\newcommand{\ca}[1]{\mathcal{#1}} %%% --"bigO"
%\newcommand{\bb}[1]{\mathbb{#1}} %%% --"Natural, Real numbers"
%\newcommand{\rom}[1]{\romannumeral{#1}} %%% --Roman numbers

%%% --Quick Arrows
\newcommand{\ra}[1]{\ifnum #1=1\rightarrow\fi\ifnum #1=2\Rightarrow\fi}

\newcommand{\la}[1]{\ifnum #1=1 \leftarrow\fi}

%%% --Special Editor Config
\renewcommand{\ni}{\noindent}
\newcommand{\onum}[1]{\raisebox{.5pt}{\textcircled{\raisebox{-1pt} {#1}}}}
\newcommand{\bbu}{\blacktriangleright}
\newcommand{\wbu}{\vartriangleright}

\newcommand{\claim}[1]{\underline{``{#1}":}}
\newcommand{\prob}[1]{\bf {#1}}

\newcommand{\bgfl}{\begin{flalign*}}
\newcommand{\bga}{\begin{align*}}
\def\beq{\begin{equation}} \def\eeq{\end{equation}}

\renewcommand{\l}{\left}\renewcommand{\r}{\right}

\newcommand{\casebrak}[2]{\left \{ \begin{array}{l} {#1}\\{#2} \end{array} \right.}
\newcommand{\ttm}[4]{\l[\begin{array}{cc}{#1}&{#2}\\{#3}&{#4}\end{array}\r]} %two-by-two-matrix
\newcommand{\tv}[2]{\l[\begin{array}{c}{#1}\\{#2}\end{array}\r]}

\newcommand{\dps}{\displaystyle}

\let\italiccorrection=\/
\def\/{\ifmmode\expandafter\frac\else\italiccorrection\fi}


%%% --General Math Symbols
\newcommand{\bc}{\because}
\newcommand{\tf}{\therefore}
\newcommand{\SUM}[2]{\sum\limits_{#1}^{#2}}
\newcommand{\PROD}[2]{\prod\limits_{#1}^{#2}}
\newcommand{\CUP}[2]{\bigcup\limits_{#1}^{#2}}
\newcommand{\CAP}[2]{\bigcap\limits_{#1}^{#2}}
\newcommand{\SUP}[1]{\sup\limits_{#1}}

\renewcommand{\o}{\circ}
\newcommand{\x}{\times}
\newcommand{\ox}{\otimes}

%%% --Special Math Characters
\newcommand{\R}{\mathbb R}%Real number
\newcommand{\N}{\mathbb N}%Nature number
\newcommand{\Z}{\mathbb Z}
\newcommand{\C}{\mathbb C}
\newcommand{\F}{\mathbb F}
\renewcommand{\O}{\mathcal{O}}
\newcommand{\A}{\mathcal{A}}%measurable sets
\renewcommand{\P}{\mathcal{P}}%power set

%%% --REAL ANALYSIS Symbols
\newcommand{\INT}[2]{\int_{#1}^{#2}}
\newcommand{\pdiff}[2]{\frac{\partial{#1}}{\partial{#2}}}
\newcommand{\UPINT}{\bar\int}
\newcommand{\UPINTRd}{\overline{\int_{\bb R ^d}}}
\newcommand{\supp}{\text{supp}}

\newcommand{\leb}{\lambda^\ast} %%% --Lebesgue
\renewcommand{\H}[1]{{\cal H}^{#1}} %%% --Hausdorff
\newcommand{\B}{\mathcal{B}} %%% --Borel set
\newcommand{\cL}{\mathcal{L}}
\newcommand{\I}{\mathcal{I}} %%% --index set
\newcommand{\Supp}[1]{\text{Supp}\left({#1}\right)}

\def\Xint#1{\mathchoice
{\XXint\displaystyle\textstyle{#1}}%
{\XXint\textstyle\scriptstyletstyle{#1}}%
{\XXint\scriptstyle\scriptscriptstyle{#1}}%
{\XXint\scriptscriptstyle\scriptscriptstyle{#1}}%
\!\int}
\def\XXint#1#2#3{{\setbox0=\hbox{$#1{#2#3}{\int}$ }
\vcenter{\hbox{$#2#3$ }}\kern-.6\wd0}}
\def\ddashint{\Xint=}
\def\dashint{\Xint-}

\def\vx{\mathbf x}

\def\lip{\left\langle}
\def\rip{\right\rangle}
\def\ega{\end{align*}}
%%%%%%%%%%%%%%%%%%%%%%%%%%%%%%%%%%%%%%%%%%%%%%%%%%%%%%%%%%%%%%%%%%%%%%%%%%%%%%%%%%%%%%%%%%%%%%%%%%%%%%%%%%%%%%%%%%%%%%%%%%%%%%%%%%%%%%%%%%%%%%%%%%%%%%%%%%%%%%%%%%%%%%%%%%%%%%%%%%%%%%%%%%%%%%%%%%%%%%
\begin{document}
\subsection*{[T1]} 
(a) Use the knowledge that $\lip u_0, u \rip = \cos\theta$, 
\bga
|\epsilon|^2 
& = \|u_0\|^2 + \lip u_0, u\rip^2 \|u\|^2 - 2\lip u_0, u\rip^2 \\
& = 1 - \cos^2\theta = \sin^2\theta
\end{align*}
(b) First note that $\lip \epsilon v, u \rip = \lip u_0 - \lip u_0, u\rip u, u \rip = \lip u_0, u\rip - \lip u_0, u\rip \|u\|^2 = 0$. Now since both $u_0$ and $u$ are normalized, we have $\lambda = \lip u , Au \rip$ and 
\bga
\lambda_0
& = \lip u_0, A u_0 \rip \\
& = \lip \epsilon v + \lip u_0, u\rip u, \epsilon A v + \lip u_0, u\rip Au\rip \\
& = \epsilon^2 \lip v, Av \rip + \lip u_0, u\rip \lip u, \epsilon Av\rip + \lip u_0, u\rip \lip \epsilon v, Au\rip + \lip u_0, u\rip^2 \lip u, Au \rip \\
& = \O(\epsilon^2) + \fgreen{2\lip u_0, u\rip \lip \epsilon v, \lambda u\rip} + \cos^2\theta \lambda \\
& = \O(\epsilon^2) + \fgreen{0} + (1-\epsilon^2)\lambda = \lambda + \O(\epsilon^2)
\end{align*}
(c) Even if $A$ is not symmetric, $\lip u_0, u\rip^2 \lambda = \lambda + \O(\epsilon^2)$ and $\lip \epsilon v, u \rip = 0$ still holds. The only problem caused by this asymmetry is $\lip u, \epsilon Av \rip \neq \lip Au, \epsilon v\rip$ and this will cause a $\O(\epsilon)$ error since we lose control to $\lip u, Av\rip$ in this case. 

\subsection*{[T2]}
(a) Observe for each Eigenpair $(\lambda_i, v_i)$, ($A v_i = \lambda_i v_i$) 
$$v_i = \lambda_i A^{-1} v_i$$
That is, $A^{-1}v_i = \/1{\lambda_i} v_i$ and $\l(\/1{\lambda_i}, v_i\r)$ is an Eigenpair of $A^{-1}$. \\
\\
(b) Similarly, consider each Eigenpair $(\lambda_i, v_i)$, write $p(x) = \SUM{j=0}r \alpha_j x^j$,
$$p(A)v_i = \SUM{j=0}r \alpha_j A^j v_i = \SUM{j=0}r \alpha_j \lambda_i^j v_i = p(\lambda_i) v_i$$
That is, $(p(\lambda_i), v_i)$ is an Eigenpair of $p(A)$. \\
\\
(c) Similarly, since $p(A) v_i = p(\lambda_i) v_i$, if $q(\lambda_i) \neq 0$,
$$\/{p(\lambda_i)}{q(\lambda_i)} q(A)  v_i = \/{p(\lambda_i)}{q(\lambda_i)}  q(\lambda_i)  v_i = p(\lambda_i)v_i$$ 
Therefore $q(A)^{-1} p(A) v_i = q(A)^{-1} p(\lambda_i) v_i = \/{p(\lambda_i)}{q(\lambda_i)}  v_i$ and $(\/{p(\lambda_i)}{q(\lambda_i)}, v_i)$ is an Eigenpair of $q(A)^{-1} p(A)$. In the other case that $q(\lambda_i) = 0$, we would find that 
$$q(A)v_i = q(\lambda_i) v_i = 0$$
That is, $q(A)$ has nontrivial null space and is not invertible! 


\subsection*{[C1]}


\end{document}