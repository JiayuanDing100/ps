\documentclass[12pt,a4paper]{article}
	%[fleqn] %%% --to make all equation left-algned--

\usepackage[top=1.2in, bottom=1.2in, left=.7in, right=.7in]{geometry}
%\usepackage{fullpage}

\usepackage{fancyhdr}\pagestyle{fancy}\rhead{Stephanie Wang}\lhead{Math245C - Homework 3}

\usepackage{amsmath,amssymb,amsthm,amsfonts,microtype,stmaryrd}
%{mathtools,wasysym,yhmath}

\usepackage[usenames,dvipsnames]{xcolor}\newcommand{\blue}[1]{\textcolor{blue}{#1}}\newcommand{\red}[1]{\textcolor{red}{#1}}\newcommand{\gray}[1]{\textcolor{gray}{#1}}
\newcommand{\fgreen}[1]{\textcolor{ForestGreen}{#1}}

\usepackage{mdframed}
	%\newtheorem{mdexample}{Example}
	%\definecolor{warmgreen}{rgb}{0.8,0.9,0.85}
	% --Example:
	% \begin{center}
	% \begin{minipage}{0.8,0.9,0.85\textwidth}
	% \begin{mdframed}[backgroundcolor=warmgreen, 
	% skipabove=4pt,skipbelow=4pt,hidealllines=true, 
	% topline=false,leftline=false,middlelinewidth=10pt, 
	% roundcorner=10pt] 
	%%%% --CONTENTS-- %%%%
	% \end{mdframed}\end{minipage}\end{center}	

%\usepackage{graphicx}
%\graphicspath{ {/Users/KoraJr/Documents/MATLAB} }
	% --Example:
	% \includegraphics[scale=0.5]{picture name}
%\usepackage{caption} %%% --some awful package to make caption...

\usepackage{hyperref}\hypersetup{linktocpage,colorlinks}\hypersetup{citecolor=blue,filecolor=blue,linkcolor=blue,urlcolor=blue}

%%% --Text Fonts
%\usepackage{times} %%% --Times New Roman for LaTeX
%\usepackage{fontspec}\setmainfont{Times New Roman} %%% --Times New Roman; XeLaTeX only

%%% --Math Fonts
%\renewcommand{\mbf}[1]{\mathbf{#1}} %%% --vector
%\newcommand{\ca}[1]{\mathcal{#1}} %%% --"bigO"
%\newcommand{\bb}[1]{\mathbb{#1}} %%% --"Natural, Real numbers"
%\newcommand{\rom}[1]{\romannumeral{#1}} %%% --Roman numbers

%%% --Quick Arrows
\newcommand{\ra}[1]{\ifnum #1=1\rightarrow\fi\ifnum #1=2\Rightarrow\fi}

\newcommand{\la}[1]{\ifnum #1=1 \leftarrow\fi}

%%% --Special Editor Config
\renewcommand{\ni}{\noindent}
\newcommand{\onum}[1]{\raisebox{.5pt}{\textcircled{\raisebox{-1pt} {#1}}}}
\newcommand{\bbu}{\blacktriangleright}
\newcommand{\wbu}{\vartriangleright}

\newcommand{\claim}[1]{\underline{``{#1}":}}
\newcommand{\prob}[1]{\bf {#1}}

\newcommand{\bgfl}{\begin{flalign*}}
\newcommand{\bga}{\begin{align*}}
\def\beq{\begin{equation}} \def\eeq{\end{equation}}

\renewcommand{\l}{\left}\renewcommand{\r}{\right}

\newcommand{\casebrak}[2]{\left \{ \begin{array}{l} {#1}\\{#2} \end{array} \right.}
%\newcommand{\ttm}[4]{\l[\begin{array}{cc}{#1}&{#2}\\{#3}&{#4}\end{array}\r]} %two-by-two-matrix
%\newcommand{\tv}[2]{\l[\begin{array}{c}{#1}\\{#2}\end{array}\r]}

\newcommand{\dps}{\displaystyle}

\let\italiccorrection=\/
\def\/{\ifmmode\expandafter\frac\else\italiccorrection\fi}


%%% --General Math Symbols
\newcommand{\bc}{\because}
\newcommand{\tf}{\therefore}
\newcommand{\SUM}[2]{\sum\limits_{#1}^{#2}}
\newcommand{\PROD}[2]{\prod\limits_{#1}^{#2}}
\newcommand{\CUP}[2]{\bigcup\limits_{#1}^{#2}}
\newcommand{\CAP}[2]{\bigcap\limits_{#1}^{#2}}
\newcommand{\SUP}[1]{\sup\limits_{#1}}

\renewcommand{\o}{\circ}
\newcommand{\x}{\times}
\newcommand{\ox}{\otimes}

%%% --Special Math Characters
\newcommand{\R}{\mathbb R}%Real number
\newcommand{\N}{\mathbb N}%Nature number
\newcommand{\Z}{\mathbb Z}
\newcommand{\C}{\mathbb C}
\renewcommand{\O}{\mathcal{O}}
\newcommand{\A}{\mathcal{A}}%measurable sets
\renewcommand{\P}{\mathcal{P}}%power set

%%% --REAL ANALYSIS Symbols
\newcommand{\INT}[2]{\int_{#1}^{#2}}
\newcommand{\pdiff}[2]{\frac{\partial{#1}}{\partial{#2}}}
\newcommand{\UPINT}{\bar\int}
\newcommand{\UPINTRd}{\overline{\int_{\bb R ^d}}}
\newcommand{\supp}{\text{supp}}

\newcommand{\leb}{\lambda^\ast} %%% --Lebesgue
\renewcommand{\H}[1]{{\cal H}^{#1}} %%% --Hausdorff
\newcommand{\B}{\mathcal{B}} %%% --Borel set
\newcommand{\cL}{\mathcal{L}}
\newcommand{\I}{\mathcal{I}} %%% --index set
\newcommand{\Supp}[1]{\text{Supp}\left({#1}\right)}

\def\Xint#1{\mathchoice
{\XXint\displaystyle\textstyle{#1}}%
{\XXint\textstyle\scriptstyle{#1}}%
{\XXint\scriptstyle\scriptscriptstyle{#1}}%
{\XXint\scriptscriptstyle\scriptscriptstyle{#1}}%
\!\int}
\def\XXint#1#2#3{{\setbox0=\hbox{$#1{#2#3}{\int}$ }
\vcenter{\hbox{$#2#3$ }}\kern-.6\wd0}}
\def\ddashint{\Xint=}
\def\dashint{\Xint-}


%%%%%%%%%%%%%%%%%%%%%%%%%%%%%%%%%%%%%%%%%%%%%%%%%%%%%%%%%%%%%%%%%%%%%%%%%%%%%%%%%%%%%%%%%%%%%%%%%%%%%%%%%%%%%%%%%%%%%%%%%%%%%%%%%%%%%%%%%%%%%%%%%%%%%%%%%%%%%%%%%%%%%%%%%%%%%%%%%%%%%%%%%%%%%%%%%%%%%%
\begin{document}
\subsection*{Problem 1 (Exercise 1.11.1. [Tao, 2011])}
\claim{$f+g$ is log-convex} \\
Refer to \texttt{http://math.stackexchange.com/questions/665768/how-to-prove-that-the-sum-of-two-log-convex-functions-is-log-convex} \\
Fix $x, y \in [0, 1]$, then from log-convexity of $f$ and $g$, there exist constants $a, b, c, d$ such that $\forall z \in [x, y]$, 
$$f(z) \leq e^{az+b} \text{ and } g(z) \leq e^{cz + d}$$
with equality holds at both endpoints. Summing both inequality we get 
$$(f+g)(z) \leq e^{az+b} + e^{cz + d}$$ with equality holds at endpoints. WLOG assume $a \geq c$, we can show now that RHS is log-convex by differentiating
\bga
\frac{d}{dx} \log \l(e^{az+b} + e^{cdz + d}\r)
& = \frac{a e^{a z+b}+c e^{c z+d}}{e^{a z+b}+e^{c z+d}} \\
& =c+\frac{a -c  }{1+e^{(c-a) z+d-b}}
\end{align*}
which is an increasing function of $z$.\\
\\
\claim{$fg$ is log-convex} Observe that 
$$\log(fg) = \log(f) + \log(g)$$
is a convex function since its sum of two convex function. \\
\\
\claim{$\max(f, g)$ is log-convex} Use notation $f_1 = f$, $f_2 = g$. Fix $x, y \in [0, 1]$, $\theta \in (0, 1)$; use monotonicity of $\log$ and the log-convexity of both functions, 
\bga
\log\max(f_1, f_2)(\theta x + (1-\theta)y) 
& \leq \log f_i(\theta x + (1-\theta)y) & \text{for } i = 1, 2\\
& \leq \theta \log f_i(x) + (1-\theta)\log f_i(y) & \text{for } i=1,2
\end{align*}\qed


\newpage\subsection*{Problem 2 (Exercise 1.11.4. [Tao, 2011])} 
Refer to \texttt{https://en.wikipedia.org/wiki/Hadamard\_three-lines\_theorem} \\
Define 
$$M(\sigma) := \sup_{t\in\R} |f(\sigma + it)|$$



\newpage\subsection*{Problem 3 (Exercise 1.11.5 [Tao, 2011])} 
Observe that the inequality is a special case of H\"older inequality, of which the equality is equivalent to the condition
$$|f|^{p_0} = C|f|^{p_1} \text{ for some constant } C$$
Now that we have $p_0 < p_1$, this is to say that $|f|^{p_1 - p_0} = C$ is a constant; however, if we applied instead H\"older inequality on a smaller domain and just sum up over the partition. \qed


\newpage\subsection*{Problem 4 (Exercise 1.11.8 [Tao, 2011])} 
The first inequality follows from Chebyshev's inequality. Now denote $X = \{x_1, \cdots, x_n\}$; to prove the second inequality, we first WLOG assume $f = \SUM{i=1}n f_i 1_{\{x_i\}}$ where $|f_n| \geq |f_{n-1}| \geq \cdots \geq |f_1| \geq 0$. Observe that 
$$\|f\|_{p, \infty} = \sup_{t>0} t\lambda_f(t)^{1/p}$$
should attain maximum at $|f_n|, \cdots, |f_2|$ or $|f_1|$, so if we normalize $\|f\|_{p, \infty} = 1$, then we know 
$$\max_{1\leq i\leq n} |f_i| (n-i+1)^{\/1p} \leq 1$$
therefore for $i=1, \cdots, n$, $f_i \leq (n-k+1)^{1/p}$. Now back to the inequality, 
\bga
\|f\|_p
& = \l(\SUM{i=1}n |f_i|^p\r)^{\/1p} \\
& \leq \l(\SUM{i=1}n (n-i+1)^{-\/1p}\r)^{\/1p} \\
& = \l(\SUM{k=1}n k^{-\/1p}\r)^{\/1p} \\
& \leq \l(\log()\r)^{\/1p} \\
& \leq 
\end{align*}



\newpage\subsection*{Problem 5 (Exercise 1.11.11 [Tao, 2011])} 
\claim{(ii) $\ra2$ (i)} WLOG assume $f \geq 0$. Take $E_t = \{x\in X: f(x) \geq t\}$, then from assumption (ii),
$$t\lambda_tf(t) \leq \int f 1_{E_t} d\mu \leq C'\lambda_f(t)^{\/1{p'}}$$
Divide both side with $\lambda_f(t)^{\/1p}$, then
$$\forall t > 0, t \lambda_f(t)^{\/1p} \leq C'$$
and taking supremum shows that $\|f\|_{L^{p, \infty}(X)} \leq C'$.\\
\\
\claim{(i) $\ra2$ (ii)} 

\newpage\subsection*{Problem 6 (Exercise 1.11.13 [Tao, 2011])} 
\claim{(i) $\ra2$ (ii)} By Exercise 1.11.11 [Tao, 2011], assuming (i) (which is the same as (i) in previous problem), there exists $C'>0$ such that for all $E\subseteq X$ with finite measure, 
$$\l|\int_X f 1_E d\mu\r| \leq C' \mu(E)^{\/1{p'}}$$
Take $E' = E$, then of course the conditions holds for (ii). \\
\\
\claim{(ii) $\ra2$ (i)} I will prove (ii) of Exercise 1.11.11 [Tao, 2011] by assuming (ii) of this problem; then by equivalence given in previous problem, that will prove (i) for this problem. 

Fix $E \subseteq X$ with finite measure, we aim to find a constant $C'$ independent of $E$ such that 
$$\l|\int f1_E d\mu\r| \leq C' \mu(E)^{\/1{p'}}$$
Denote $E_0 = E$. Take from the assumption that there exists $E_1 \subseteq E_0$ with $\mu(E_1) \geq \/12\mu(E)$ and 
$$\l|\int f1_{E_1} d\mu\r| \leq C\mu(E)^{\/1{p'}}$$
and successively take $E_2, E_3, \cdots$ and denote $F_k = \CUP{i=1}k E_i$ such that $E_{k+1} \subseteq E \setminus F_k$ (therefore $E_i$'s are disjoint) with 
$$\mu(E_{k+1}) \geq \/12\mu(E \setminus F_k)$$
and 
$$\l|\int f 1_{E_{k+1}} d\mu\r| \leq C\mu(E\setminus F_k)^{\/1{p'}}$$
With these conditions we see that $\mu(E \setminus F_k) \leq \/1{2^k} \mu(E)$ and therefore 
\bga
\l|\int f 1_{F_k} d\mu\r|
& \leq \SUM{i=1}k \l|\int f1_{E_i} d\mu\r|\\
& \leq \SUM{i=1}k \/C{2^{k}}\mu(E)^{\/1{p'}} \leq 2C\mu(E)
\end{align*}
Exhaust $k \ra1 \infty$ then we know $F_k \nearrow E$ and therefore by DCT $\int f1_{F_k} d\mu \ra1 \int f1_E d\mu$ and the bound from RHS remains solid
$$\l|\int f1_E d\mu \r| \leq 2C \mu(E)^{\/1{p'}}$$
\end{document}


\claim{if} The equality certainly holds if $f$ is a constant times a characteristic function; moreover, we can separate the domain and consider the integral separately. Note that 
$$\frac{p_\theta}{p_0} = \l(\frac{p_1}{p_0}\r)^\theta, \frac{p_\theta}{p_1} = \l(\frac{p_0}{p_1}\r)^{1-\theta}$$
Consider $f = a1_E + b1_F$, 
\bga
\|f\|_{L^{p_\theta}(X)}^{p_\theta}
& = a^{p_\theta} \mu(E) + b^{p_\theta} \mu(F)\\
\|f\|^{(1-\theta)p_\theta}_{L^{p_0}(X)}\|f\|^{\theta p_\theta}_{L^{p_1}(X)}
& = \l(a^{p_0}\mu(E) + b^{p_0} \mu(F)\r)^{(1-\theta)\frac{p_\theta}{p_0}}\l(a^{p_1}\mu(E) + b^{p_1} \mu(F)\r)^{\theta\frac{p_\theta}{p_1}} \\
& = \l(a^{p_0}\mu(E) + b^{p_0} \mu(F)\r)^{(1-\theta)\l({p_1}/{p_0}\r)^\theta}\l(a^{p_1}\mu(E) + b^{p_1} \mu(F)\r)^{\theta\l({p_0}/{p_1}\r)^{1-\theta}} \\
\end{align*} 